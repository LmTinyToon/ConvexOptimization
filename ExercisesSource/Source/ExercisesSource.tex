% Note: this file was compiled via TeXnicCenter program with following compiler keys (changed by myself)
%   -interaction=nonstopmode "%wm" -aux-directory="../Temp" --output-directory="../Bin"
% Note: my viewer (stdu viewer) uses following key
%    Viewer project's output "../Bin/ExercisesSource.pdf"

\documentclass[a4paper, 12pt]{article}
\begin{document}	
TODO: (alex) think about proper decoration and styling of document!
\section{Convex sets}
%		Exercise 2.1 about convex set definition
Exercise 2.1 Proof.

We should prove generalization of convexity definition. We will use mathematical induction to prove this statement

1) Base case, where $n=2$. It is obvious by definition of convex set.

2) Suppose, that statement is correct for any $n>=2$

3) We will prove statement for $n+1$. Let $\theta_i i \in R, i \in 1,...,n+1$, 
with $\theta_1 + ... + \theta_{n+1} = 1, \theta_i >= 0$. Also $x_i \in R^n$. Then we can use following fact (we suppose, that $\theta_{n+a} <> 1$): 
$$\frac{\theta_1}{1-\theta_{n+1}} + ... + \frac{\theta_n}{1-\theta_{n+1}} = 1$$

Note, that this fact will lead from initial requirement for $\theta_i$.  
$$\theta_1 x_1 + ... + \theta_{n+1} x_{n+1} = 
(1-\theta_{n+1})(\frac{\theta_1}{1 - \theta_{n+1}} x_1 + ... + \frac{\theta_n}{1 - \theta_{n+1}} x_n) + 
\theta_{n+1} x_{n+1} = (1-\theta_{n+1}) x_c + \theta_{n+1} x_{n+1} \in C$$

$x_c \in C due *$
%		ex. end

\end{document}