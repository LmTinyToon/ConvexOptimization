%		Exercise 2.1
Exercise 2.1 Proof.

We should prove generalization of convexity definition. We will use mathematical induction to prove this statement

1) Base case, where $n=2$. It is obvious by definition of convex set.

2) Suppose, that statement is correct for any $n>=2$

3) We will prove statement for $n+1$. Let $\theta_i i \in R, i \in 1,...,n+1$, 
with $\theta_1 + ... + \theta_{n+1} = 1, \theta_i >= 0$. Also $x_i \in R^n$. Then we can use following fact (we suppose, that $\theta_{n+a} <> 1$): 
$$\frac{\theta_1}{1-\theta_{n+1}} + ... + \frac{\theta_n}{1-\theta_{n+1}} = 1$$

Note, that this fact will lead from initial requirement for $\theta_i$.  
$$\theta_1 x_1 + ... + \theta_{n+1} x_{n+1} = 
(1-\theta_{n+1})(\frac{\theta_1}{1 - \theta_{n+1}} x_1 + ... + \frac{\theta_n}{1 - \theta_{n+1}} x_n) + 
\theta_{n+1} x_{n+1} = (1-\theta_{n+1}) x_c + \theta_{n+1} x_{n+1} \in C$$

$x_c \in C due *$
%		Exercise 2.1 end

%		Exercise 2.4
Exercise 2.4 Proof

We should prove that $conv C \subseteq \bigcup _{i=1}^{\infty} S_{i}$, where $C$ is any set, $S_{i}$
is any convex set, containing $C$. And $conv C$ is convex hull of $C$.

Note that statement $A = B$, where $A, B$ are any sets means that $A \subseteq B \subseteq A$. We will use this one for proof. Let $S = \bigcup _{i=1}^{\infty} S_{i}$.

1) Prove that $conv C \subseteq S$. Suppose opposite. Let $\exists x_{0} \in conv C : x_{0} \notin S$.
It means that $\exists \theta_1, \theta_2 \in R : \theta_i >= 0, \theta_1 + theta_2 = 1$ and $\exists x_1, x_2 \in C | x_1 \theta_1 + x_2 \theta_2 = x_0 \in conv C$. Set $S$ is convex and $C \subseteq S$. It yields that $\theta_1 x_1 + \theta_2 x_2 = x_0 \in S$. It is contradiction.

2) Prove that $S \subseteq conv C$. Suppose opposite. Let $\exists x_0 \in S : x0 \notin conv C, x_0 \in S$. Thus $x0 \in S_i \forall i$. We know that $C \subseteq conv C$ and $conv C$ is convex. Hence $\exists j : S_j = conv C$ and $x_0 \in S_j = conv C$. It is contradiction.

%		Exercise 2.4 end